\section{report on Use of AI}

1.

OpenAI ChatGPT (Feb 2,2024 version,ChatGPT3.5)

Query1:  \emph{
    “swings in play and runs of success by one player are random” 
can this sentence to be interpreted as  swings in play and runs of success by one player are independent?
}

Output: \emph{Yes, the sentence "swings in play and runs of success by one player are random" 
can be interpreted as suggesting that swings in play and runs of success by one player are 
independent of each other. In statistical terms, independence means that the occurrence or 
outcome of one event does not influence the occurrence or outcome of another.}

\emph{
In the context of tennis, this would imply that the fluctuations in the dynamics of the match 
(swings in play) and a player's success streaks (runs of success) are not connected or dependent 
on each other. Each event is considered random and not influenced by the occurrence of the other.
}
\\ \hspace*{\fill} \\
2.

OpenAI ChatGPT (Feb 5,2024 version,ChatGPT3.5)

Query2:
\emph{
    what is general technical statistics in table tennis games
}

Output:
\emph{In table tennis, general technical statistics refer to the analysis 
and measurement of various technical aspects of a player's performance during a match. 
These statistics provide valuable insights into a player's strengths, weaknesses, 
and overall playing style. Here are some general technical statistics commonly used 
in table tennis games:}

\emph{Serve Statistics:}

\emph{
    Serve Type: Recording the types of serves a player employs, such as forehand, backhand, short, long, or sidespin serves.
Serve Effectiveness: Analyzing the success rate of different types of serves, including direct points and opponent errors.
}
\emph{Receiving Statistics:
Receiving Techniques: Documenting how a player receives various types of serves, such as forehand push, backhand push, flick, or topspin.
Receiving Success Rate: Calculating the player's success rate in returning different types of serves.
Stroke Execution Statistics:
Forehand and Backhand Shots: Analyzing the usage and effectiveness of forehand and backhand strokes in various situations.
Types of Strokes: Recording the diversity of strokes played, including topspin, backspin, drives, and smashes.
Rally Statistics:
Rally Length: Measuring the average length of rallies in terms of the number of strokes exchanged.
Rally Outcome: Tracking the outcomes of rallies, such as points won or lost, and identifying patterns in rally progression.
Placement and Direction Statistics:
Ball Placement: Analyzing where a player tends to place the ball on the opponent's side of the table.
Direction of Shots: Recording the direction of shots, including cross-court, down the line, and shots to the opponent's body.
Point Distribution Statistics:
Point Sources: Identifying the sources of points, such as winners, opponent errors, or forced errors.
Point Distribution: Analyzing how points are distributed throughout the match, helping to identify key moments.
Error Analysis:
Unforced Errors: Documenting the number of unforced errors made by a player.
Forced Errors: Identifying instances where the opponent's play led to errors by the player.
These technical statistics are crucial for players, coaches, and analysts to assess performance, 
strategize for future matches, and tailor training programs to address specific areas of improvement. 
They provide a comprehensive overview of a player's game and contribute to a more informed and strategic 
approach to table tennis.}