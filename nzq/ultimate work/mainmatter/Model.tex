\section{Momentum Evaluation Model}

To determine which player is performing better at a specific time, we create a indicator ``Momentum'' 
to give a quantitative and overall evaluation.

\begin{definition}
    Niche width is the range of resources that a species can use.
\end{definition}
Niche width is an indicator \cite{Alice13}

\subsection{Model Overview}

\subsection{Data Processing and Normalization}
In order to quantify the factors used in our model, based on our assumptions, we calculate them using the following formulae: \par
\begin{equation}
    P_{ace} = \frac{\sum_{p \in S_3} p_{ace}}{3}
\end{equation}
\begin{equation}
    P_{df} = -\frac{\sum_{p \in S_3} p_{double\_fault}}{3}
\end{equation}
\begin{equation}
    P_{1st} = \frac{\sum_{p \in S_3} [p_{serve\_no} = 1]}{3}
\end{equation}
\begin{equation}
    P_{fw} = \frac{\sum_{p \in S_3} [p_{rally\_count} \le 3] [p_{point\_victor} = player]}{3}
\end{equation}
\begin{equation}
    rd = \frac{\sum_{p \in R_3} \left\{
        \begin{aligned}
        0, && p_{return\_depth} = ND \\
        1, && p_{return\_depth} = D \\
        -1, && p_{return\_depth} = NA \\
        \end{aligned}
        \right.}{3}
\end{equation}
\begin{equation}
    P_{win} = \frac{\sum_{p \in H_3} p_{winner}}{3}
\end{equation}
\begin{equation}
    P_{net} = \frac{\sum_{p \in H_3} p_{net\_pt\_won}}{\sum_{p \in H_3} p_{net\_pt}}
\end{equation}
\begin{equation}
    dist = \left\{
        \begin{aligned}
        0, && point_{cur, distance\_run} < 5 \\
        -1, && point_{cur, distance\_run} > 45 \\
        \frac{5 - point_{cur, distance\_run}}{40}, && otherwise \\
        \end{aligned}
        \right.
\end{equation}
\begin{equation}
    P_{unf} = -\frac{\sum_{p \in H_3} p_{unf\_err}}{3}
\end{equation}
\begin{equation}
    scored = [point_{cur, point\_victor} = player]
\end{equation}
\begin{equation}
    diff = \frac{\sum_{p \in point}[p_{set\_no} = point_{cur, set\_no}][p_{game\_no} = point_{cur, game\_no}](2[p_{point\_victor} = player]-1)}{\min\{3, \sum_{p \in point}[p_{set\_no} = point_{cur, set\_no}][p_{game\_no} = point_{cur, game\_no}]\}}
\end{equation}
\par In order to normalize the data processed, we convert the original data to limit them in $[-1, 1]$. For those factors that negatively influence the momentum, such as $P_{df}$, we made sure it's in $[-1, 0]$. For those factors that positively influence the momentum, such as $P_{win}$, we made sure it's in $[0, 1]$. For those factors that influence the momentum in both ways, such as $diff$, we made sure it's in $[-1, 1]$.

\begin{algorithm}
    \caption{An algorithm with caption}\label{alg:cap}
    \begin{algorithmic}
    \Require $n \geq 0$
    \Ensure $y = x^n$
    \State $y \gets 1$
    \State $X \gets x$
    \State $N \gets n$
    \While{$N \neq 0$}
    \If{$N$ is even}
        \State $X \gets X \times X$
        \State $N \gets \frac{N}{2}$  \Comment{This is a comment}
    \ElsIf{$N$ is odd}
        \State $y \gets y \times X$
        \State $N \gets N - 1$
    \EndIf
    \EndWhile
    \end{algorithmic}
\end{algorithm}

\subsection{Visualization and Analysis}

In figure 1, when the red line is above the blue line, it means that the player is performing better than the opponent. 

In figure 2, we minus the opponent's momentum from the player's momentum to get the difference. And the difference 
indicates how much better the player is performing than the opponent.

\subsection{momentum autocorrelation and correlation with runs of success}

to 

\paragraph{momentum autocorrelation}

\paragraph{correlation with runs of success}

To give a quantitative evaluation of ``future scores'', we calculate points gain in future multiple points,
and calculate difference between the two players, 
this indicates how much better the player is performing than the opponent.