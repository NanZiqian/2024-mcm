\section{Robustness Analysis}

\subsection{Generalization ability}

In the previous section, we have only used the data of the first three matches. In this section we will focus on the generlization of the model.

\paragraph{test on all given data}~{}

We use our model in problem 3 in predicting all matches to verify its accuracy. Considering the difference length of the matches, we use weighted
accuracy to denote the generability of our model. The fomula is as follows:
$$ P_{avg}=\frac{1}{L_{tot}}\sum\limits_{matches}\frac{P_i}{L_i}$$
where $L_i$ stands for the length of the match, we measure it by point number.

So the general accuracy of our model in predicting matches is //////, \\
following are some factors that are not included in our model, 
all of which we think may influence the result, some of them may be hard to quantify, some of them are not included due to lack
of data. Adding them to the model can make it more complete, this is left for future work.
\begin{enumerate}
    \item The change of the players' strategies, during the course of the game, they may become more 
    familiar with the opponent's technical characteristics and make targeted changes to shift "momentum".
    \item The possibility that the players intentionally hide their strength, this is similar to the previous item.
    \item The incentive of the audience, this is more likely to have something to do with the difference of home and away,
    just like football. The influence of the audience on foreign and native players is totally different.
\end{enumerate}

Now, we have finished Problem 4.

\subsection{Parameters Sensitivity Analysis}