\thispagestyle{empty}
\vspace*{-16ex}
\centerline{\begin{tabular}{*3{c}}
	\parbox[t]{0.3\linewidth}{\begin{center}\textbf{Problem Chosen}\\ \Large \textcolor{red}{\Problem}\end{center}}
	& \parbox[t]{0.3\linewidth}{\begin{center}\textbf{2024\\ MCM/ICM\\ Summary Sheet}\end{center}}
	& \parbox[t]{0.3\linewidth}{\begin{center}\textbf{Team Control Number}\\ \Large \textcolor{red}{\Team}\end{center}}	\\
	\hline
\end{tabular}}

\begin{center}
	\Large \textbf{Shortest Path Algorithms:~Taxonomy and Advance in Research}
\end{center}

Every athlete understand the importance of practicing skills and strategies to win a match.
But in most sports, they are not the only determining factors of the outcome of a match.
A young and fearless athlete may surprisingly defeat a veteran, or
even the most acknowledged and well-known athlete may lose a match due to a momentary lapse in concentration
and the tide it raises. Sometimes the outcome seems destined to be altered by a force, even multiple times.
We call it momentum.\cite{Alice13} Such is the charm of sports, and such is the charm of tennis.
Out of curiosity, we are interested in studying the momentum, a mysterious force in tennis
with modern methods.

As for problem 1, we select potential influencing factors and use Analytic Hierarchy Process (AHP) method 
to make a preliminary sort of importance of the influencing factors. We find that the three most influential factor is 
score difference, whether scored in the last point and running distance.

In problem 2, we perform autocorrelation test on momentum, 
and we find that it has strong first-order autocorrelation with a corrcoef of approximately 0.5149. Then we quantify 
the future scores and then determine the correlation between momentum and future scores, they are highly correlated,
and with a maximum correlation coefficient of 0.7934 with next score.

In problem 3, to predict the swings in the match , we establish model based on Gated Recurrent Unit(GRU) network.
We process data in problem 1 and add more features. We define the swings and use previous information to predict future.
Then compared to the result in problem 1, our accuracy is ....\\

To identify the most related factors, we use a novel argorithm called Permutation Feature Importance argorithm, 
it can determine the importance of features by calculating their prediction error after permutation. We find that ....\\

As for the ideas for specific atheletes, we use previous model on their matches to identify their feature importance.
We find that different atheletes have different features in their match.\\

Problem 4\\

Key words: AHP, correlation analysis, GRU, Permutation Feature Importance

Indroduction\\
Tennis more than any other sport, is a game of momentum. 
The absence of a clock to do the dirty work of finishing off an opponent, 
and a scoring system basedon units used, makes the flow of the match much more important than any 
lead that has been established.--Chuck Kriese\\