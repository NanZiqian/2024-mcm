\thispagestyle{empty}
\vspace*{-16ex}
\centerline{\begin{tabular}{*3{c}}
	\parbox[t]{0.3\linewidth}{\begin{center}\textbf{Problem Chosen}\\ \Large \textcolor{red}{\Problem}\end{center}}
	& \parbox[t]{0.3\linewidth}{\begin{center}\textbf{2024\\ MCM/ICM\\ Summary Sheet}\end{center}}
	& \parbox[t]{0.3\linewidth}{\begin{center}\textbf{Team Control Number}\\ \Large \textcolor{red}{\Team}\end{center}}	\\
	\hline
\end{tabular}}

\begin{center}
	\Large \textbf{Unravel the Certainty in the Mysterious Momentum: An application of Data Science in Sports Analysis}
\end{center}

Every athlete understand the importance of practicing skills and strategies to win a match.
But in most sports, they are not the only determining factors of winning.
A young and fearless athlete may surprisingly defeat a veteran, or
even the most acknowledged and well-known athlete may lose a match due to a momentary lapse 
thus triggering a turn over. Sometimes the outcome seems destined to be altered by a force, even multiple times.
We call it the momentum. Such is the charm of sports, and such is the charm of tennis.
Out of curiosity, we are interested in studying the momentum, a mysterious force in tennis,
with modern data science methods, trying to figure out the certain factors that affect momentum.

As for problem 1, we selected factors that reflect player's performance and used Analytic Hierarchy Process (AHP) method 
to calculate the weights. We found that the three most influential factor are
score difference, whether scored in the last point and running distance. The sum of the weights multiplied by factors
reflecting the performance is referred to as momentum. 

In problem 2, we performed autocorrelation test on momentum, 
and we found that it has strong first-order autocorrelation (0.5149). Then we quantified
the future scores and determined the correlation between momentum and future scores, they are highly correlated.
Momentum has a maximum correlation coefficient of 0.7934 with next score.

In problem 3, to predict the swings in the match, we established classification model based on Gated Recurrent Unit (GRU),
a type of neural network.
We first defined the swings and classified them into four patterns, so the future swing could be classified into
four categories. Then we processed data that's suitable for training, and trained different model out of different purposes.

We trained model with data with and without recent scores, to figure out recent scores' impact on future swings.
We trained model with data of a single player, to give customized advice.
And we trained model with data of all matches, to figure out general factors that affect future swings of all players.
We found that the accuracy on the test data of all models is around 0.4274, better than random guess of the four patterns,
which implies that there're determining factors of swings in our model.

To identify the most relevant factors, we used a novel algorithm called Permutation Feature Importance algorithm.
It can determine the importance of features of input in neural network. We found that the most important feature while
the data includes recent score is still the ``score difference'', and without, is ``whether win within 3 rally count'',
from which we inferred that recent scores has significant impact on momentum,
improving our understanding of problem 2.

As for the advice for athletes, we trained model on their matches to identify their feature importance,
so we could provide them with customized advice.

In problem 4, we chose model trained on half of the data and tested on the other half, and the accuracy is around 0.4115.
As for women's tennis and table tennis, sadly we failed to find abundant data to train the model.
But the path forward after finding the data is clear: we reassess AHP and train the GRU network.

The letter, including advice for coaches and athletes, is attached in the appendix.

Key words: Momentum in Tennis, AHP, Correlation Analysis, GRU, Permutation Feature Importance